\documentclass{beamer}

\usepackage{kclecture}
\usepackage[italic,defaultmathsizes,eulergreek]{mathastext}
\usepackage{proof}
\usepackage{mathtools}

\begin{document}

\title{Formalizing the meta-theory of focusing\thanks{Joint work with:
    Giselle Reis, Leonardo Lima}}

\author{
  {\larger\textbf{Kaustuv Chaudhuri}}
  \\
  Inria \& LIX
}
\date{
  2018-02-28
}

\begin{frame}
  \maketitle
\end{frame}

\begin{frame}
  \frametitle{Context}

  \begin{itemize}
    \itemsep 1em
  \item Project: formalizing deductive \& computational systems
  \item Encode: formulas, sequents, proof systems
  \item Prove: meta-theorems (think cut-admissibility)
  \item Tool: Abella (\url{http://abella-prover.org})
  \end{itemize}
\end{frame}

\begin{frame}
  \frametitle{Why formalize meta-theory?}

  \begin{itemize}[<+->]\itemsep 1em
  \item Proof theory abounds with formalisms
    \begin{itemize}[<.->]
    \item[---] natural deduction
    \item[---] \{shallow, nested, labeled, hyper\} sequents
    \item[---] calculus of structures
    \item[---] subatomic, combinatorial, \dots
    \end{itemize}
  \item Many of our main results are syntactic
    \begin{itemize}[<.->]
    \item[---] cut-elimination
    \item[---] invertibility, permutations, absorption, \dots
    \item[---] focusing
    \item[---] decidability, complexity, \dots
    \end{itemize}
  \item Syntactic proofs are kind of tedious
    \begin{itemize}[<.->]
    \item[---] induction
    \item[---] large numbers of cases
    \item[---] many cases similar
    \item[---] detail $\longleftrightarrow$ comprehension
    \end{itemize}
  \end{itemize}
\end{frame}

% \begin{frame}
%   \frametitle{Anecdata}
%
%   \begin{itemize}
%   \item Girard's ``proof'' of cut-elimination for linear logic
%     \begin{itemize}
%     \item[---] Fixed by Danos in his PhD
%     \end{itemize}
%   \item Bierman's ``proof'' of cut-elimination for full intuitionistic
%     linear logic (FILL)
%     \begin{itemize}
%     \item[---] Fixed by Bra\"uner and de Paiva
%     \end{itemize}
%   \item Valentini's ``proof'' of cut-elimination for provability logic
%     (GL) in the system GLS$_V$
%     \begin{itemize}
%     \item[---] Fixed by Gor\'e and Ramanayake
%     \item[---] Formalized in Isabelle/HOL by Dawson and Gor\'e
%     \end{itemize}
%   \item A number of incorrect ``proofs'' of cut-elimination for
%     bi-intuitionistic logic
%     \begin{itemize}
%     \item[---] Fixed by Pinto and Uustalu
%     \end{itemize}
%   \item A ``proof'' of cut-free completeness for a certain formulation
%     of nested sequents for the intuitionistic modal logic KD by
%     Stra{\ss}burger and Marin
%     \begin{itemize}
%     \item[---] So far only sidestepped, not fixed
%     \end{itemize}
%   \end{itemize}
% \end{frame}

% \begin{frame}
%   \frametitle{The \alert<2->{computer} science solution}
%
%   \pause
%
%   \judg{Let's have a \alert{computer} check our work.}
%
%   \bigskip \pause
%
%   Benefits
%   \begin{itemize}
%   \item Computers don't get tired
%   \item Computers don't get bored
%   \item Computers don't forget stuff
%   \end{itemize}
%
%   \bigskip \pause
%
%   Drawbacks
%   \begin{itemize}
%   \item Computers are kind of dumb
%   \item Computers don't speak our language
%   \item Computers require excessive detail
%   \end{itemize}
% \end{frame}

\begin{frame}
  \frametitle{A sequent calculus for linear logic}

  \onslide<2->{Classical, propositional, one-sided, multiplicative-additive}
  \begin{gather*}
    \linfer{
      \ts a, \dual a
    }{}
    \qquad
    \linfer{
      \ts \G, \D, A \TENS B
    }{
      \ts \G, A
      &
      \ts \D, B
    }
    \qquad
    \linfer{
      \ts \ONE
    }{}
    \qquad
    \linfer{
      \ts \G, A \PAR B
    }{
      \ts \G, A, B
    }
    \qquad
    \linfer{
      \ts \G, \BOT
    }{
      \ts \G
    }
    \\[2ex]
    \linfer{
      \ts \G, A \WITH B
    }{
      \ts \G, A
      &
      \ts \G, B
    }
    \qquad
    \linfer{
      \ts \G, \TOP
    }{}
    \qquad
    \linfer{
      \ts \G, A \PLUS B
    }{
      \ts \G, A
    }
    \qquad
    \linfer{
      \ts \G, A \PLUS B
    }{
      \ts \G, B
    }
  \end{gather*}

  \pause\pause

  \judg{
    \infer=[cut]{
      \ts \G, \D
    }{
      \ts \G, A
      &
      \ts \D, \dual A
    }
  }

\end{frame}


\begin{frame}
  \frametitle{Commutative cases}

  \begin{multline*}
    \linfer={
      \ts \G_1, \G_2, B \TENS C, D
    }{
      \infer{
        \ts \G_1, \G_2, B \TENS C, A
      }{
        \deduce{
          \ts \G_1, B
        }{\DD_1}
        &
        \deduce{
          \ts \G_2, C, A
        }{\DD_2}
      }
      &
      \deduce{
        \ts \D, \dual A
      }{\EE}
    }
    \quad \rightsquigarrow \quad
    \infer{
      \ts \G_1, \G_2, B \TENS C, \D
    }{
      \deduce{
        \ts \G_1, B
      }{\DD_1}
      &
      \infer={
        \ts \G_2, C, \D
      }{
        \deduce{
          \ts \G_2, C, A
        }{\DD_2}
        &
        \deduce{
          \ts \D, \dual A
        }{\EE}
      }
    }
  \end{multline*}
\end{frame}

\begin{frame}
  \frametitle{Principal cases}

  \begin{gather*}
    \linfer={
      \ts \G_1, \G_2, \D
    }{
      \linfer{
        \ts \G_1, \G_2, A \TENS B
      }{
        \deduce{\ts \G_1, A}{\DD_1}
        &
        \deduce{\ts \G_2, B}{\DD_2}
      }
      &
      \linfer{
        \ts \D, \dual A \PAR \dual B
      }{
        \deduce{\ts \D, \dual A, \dual B}{\EE}
      }
    }
    \quad \rightsquigarrow \quad
    \infer={
      \ts \G_1, \G_2, \D
    }{
      \deduce{\ts \G_1, A}{\DD_1}
      &
      \infer={
        \ts \G_2, \D, \dual A
      }{
        \deduce{\ts \G_2, B}{\DD_2}
        &
        \deduce{\ts \D, \dual A, \dual B}{\EE}
      }
    }
  \end{gather*}
\end{frame}


\begin{frame}
  \frametitle{Principal cases \alert{-- or maybe}}

  \begin{multline*}
    \linfer={
      \ts \G_1, \G_2, \D
    }{
      \infer{
        \ts \G_1, \G_2, A \TENS B
      }{
        \deduce{\ts \G_1, A}{\DD_1}
        &
        \deduce{\ts \G_2, B}{\DD_2}
      }
      &
\alert{
      \deduce{
        \ts \D, \dual A \PAR \dual B
      }{\EE}
}
    }
    \quad \rightsquigarrow\\
    \infer={
      \ts \G_1, \G_2, \D
    }{
      \deduce{\ts \G_1, A}{\DD_1}
      &
      \infer={
        \ts \G_2, \D, \dual A
      }{
        \deduce{\ts \G_2, B}{\DD_2}
        &
 \alert{
       \infer[$\PAR^{-1}$]{
         \ts \D, \dual A, \dual B
       }{
         \deduce{
           \ts \D, \dual A \PAR \dual B
         }{\EE}
       }
}
      }
    }
  \end{multline*}
\end{frame}

\begin{frame}[t]
  \frametitle{Lexicographic induction}

  \begin{align*}
    P, Q, \dotsc &::=
    a \mid A \TENS B \mid \ONE \mid
    A \PLUS B \mid \ZERO \\
    N, M, \dotsc &::=
    \dual a \mid A \PAR B \mid \BOT \mid
    A \WITH B \mid \TOP
  \end{align*}

  \pause

  \judg{
    \infer={
      \ts \G, \D
    }{
      \deduce{\ts \G, P}{\DD}
      &
      \ts \D, \dual P
    }
  }

  \pause

  The inductive hypothesis may be used whenever:
  \begin{enumerate}
  \item either the size of $P$ (= rank) is smaller, or
  \item the rank is the same and the height of $\DD$ is smaller.
  \end{enumerate}
\end{frame}

\begin{frame}
  \frametitle{Encoding an inference rule}

  \larger

  \judg{
    \infer{
      \ts \G, \D, A \TENS B
    }{
      \ts \G, A
      &
      \ts \D, B
    }
  }

  \pause

  \begin{gather*}
    \infer{
      \ts (\G', \D'), A \TENS B
    }{
      \ts \G
      &
      (\G = \G', A)
      &
      \ts \D
      &
      (\D = \D', B)
    }
  \end{gather*}

  \pause

  \begin{gather*}
    \infer{
      \ts ((\G \setminus A), (\D \setminus B)), A \TENS B
    }{
      \ts \G
      &
      \qquad
      \ts \D
    }
  \end{gather*}
\end{frame}

\begin{frame}
  \frametitle{Principal cases}

  \judg{
    \smaller
    \infer{
      \ts (\G \setminus A), (\D \setminus \dual A)
    }{
      \ts \G
      &
      \qquad
      \ts \D
    }

  }

  \begin{multline*}
    \linfer={
      \ts
      ((\G_1\setminus A), (\G_2\setminus B)),
      (\D \setminus \dual A \PAR \dual B)
    }{
      \infer{
        \ts ((\G_1\setminus A), (\G_2\setminus B)),
        A \TENS B
      }{
        \deduce{\ts \G_1}{\DD_1}
        &
        \qquad
        \deduce{\ts \G_2}{\DD_2}
      }
      &
      \deduce{
        \ts \D
      }{\EE}
    }
    \quad \rightsquigarrow\\
    \infer[\alert{$=$}]{
      \alert{
        \ts
        ((\G_1 \setminus A),
        (\G_2 \setminus B)),
        (\D \setminus \dual A \PAR \dual B)
      }
    }{
    \infer={
      \ts
      (\G_1 \setminus A),
      ((\G_2 \setminus B),
      (\D \setminus \dual A \PAR \dual B))
    }{
      \deduce{\ts \G_1}{\DD_1}
      &
      \infer[\alert{$=$}]{
        \alert{
          \ts ((\G_2\setminus B),
          (\D\setminus\dual A \PAR \dual B)), \dual A
        }
      }{
      \infer={
        \ts (\G_2\setminus B),
        ((\D\setminus\dual A \PAR \dual B), \dual A)
      }{
        \deduce{\ts \G_2}{\DD_2}
        &
       \infer[$\PAR^{-1}$]{
         \ts
         ((\D\setminus \dual A \PAR \dual B),
         \dual A), \dual B
       }{
         \deduce{
           \ts \D
         }{\EE}
       }
      }}
    }}
  \end{multline*}
\end{frame}

\begin{frame}
  \frametitle{A multiset theory}

  \begin{itemize}
  \item $\G, \D = \D, \G$
  \item $(\G, \D), \W = (\G, \D), \W$ \pause
  \item $(\G \setminus A) \setminus B = (\G \setminus B) \setminus A$ \pause
  \item $(\G,\D) \setminus A =
    \begin{cases}
      (\G\setminus A), \D \\
      \G, (\D\setminus A)
    \end{cases}$
  \end{itemize}
\end{frame}

\begin{frame}[fragile]
  \frametitle{Formalizing multisets}

  \begin{lstlisting}
Define adj : olist -> o -> olist -> prop by
  adj nil A (A :: nil) ;
  adj (B :: L) A (B :: K) := adj L A K.

Define merge : olist -> olist -> olist -> prop by
  merge nil nil nil ;
  merge J K L :=
    exists A JJ LL, adj JJ A J /\ adj LL A L
      /\ merge JJ K LL ;
  merge J K L :=
    exists A KK LL, adj KK A K /\ adj LL A L
      /\ merge J KK LL.

Define perm : olist -> olist -> prop by
  perm J K := merge J nil K.
  \end{lstlisting}
\end{frame}

\begin{frame}[fragile]
  \frametitle{Multiset properties}

  \begin{itemize}
  \item $(\G, \D), \W = \G, (\D, \W)$
  \end{itemize}

\begin{lstlisting}
Theorem merge_assoc : forall G D W GD DW GDW,
  merge G D GD -> merge GD W GDW ->
  merge D W DW -> merge G DW GDW.

Theorem merge_assoc_perm : forall G D W GD DW U V,
  merge G D GD -> merge GD W U ->
  merge D W DW -> merge G DW V ->
  perm U V.
\end{lstlisting}

\end{frame}

\begin{frame}[fragile,fragile,fragile]
  \frametitle{Multiset properties}

  \begin{itemize}
  \item $(\G \setminus A) \setminus B = (\G \setminus B) \setminus A$
  \end{itemize}

\begin{lstlisting}
Theorem adj_reorder : forall G A D B W,
  adj G A D -> adj D B W ->
  exists U, adj G B U /\ adj U A W.
\end{lstlisting}

  \pause

  \begin{itemize}
  \item $(\G,\D) \setminus A =
    \begin{cases}
      (\G\setminus A), \D \\
      \G, (\D\setminus A)
    \end{cases}$
  \end{itemize}

\begin{lstlisting}
Theorem merge_unadj : forall G D W A U,
  merge G D W -> adj U A W ->
     (exists GG, adj GG A G /\ merge GG D U)
  \/ (exists DD, adj DD A D /\ merge G DD U).
\end{lstlisting}

\end{frame}

% \begin{frame}[fragile]
%   \frametitle{More properties}
%
% \begin{lstlisting}
% Theorem adj_same_result : forall G D A W,
%   adj G A W -> adj D A W -> perm G D.
%
% Theorem adj_diamond : forall G A D B W,
%   adj G A W -> adj D B W ->
%      (A = B /\ perm G D)
%   \/ (exists U, adj U B G /\ adj U A D).
% \end{lstlisting}
% \end{frame}

\begin{frame}[fragile]
  \frametitle{Formalizing Formulas}

\begin{lstlisting}
Kind atm type.

Type atom, natom   atm -> o.
Type tens, par     o -> o -> o.
Type one,  bot     o.
Type wth,  plus    o -> o -> o.
Type top,  zero    o.

Define dual : o -> o -> prop by
; dual (atom A) (natom A)
; dual (tens A B) (par AA BB) := dual A AA /\ dual B BB
; dual one bot
; dual (plus A B) (wth AA BB) := dual A AA /\ dual B BB
; dual zero top.
\end{lstlisting}
\end{frame}

\begin{frame}[fragile]
  \frametitle{Formalizing MALL}

  \mbox{} \bigskip

\begin{lstlisting}
Define mall : olist -> prop by
  % init
; mall L :=
    exists A, adj (natom A :: nil) (atom A) L

  % tensor
; mall L :=
    exists A B LL, adj LL (tens A B) L /\
    exists JJ KK, merge JJ KK LL /\
    exists J, adj JJ A J /\ mall J /\
    exists K, adj KK B K /\ mall K.

; mall (one :: nil)

  % par
; mall L :=
    exists A B LL, adj LL (par A B) L /\
    exists J, adj LL A J /\
    exists K, adj J B K /\
    mall K

; mall L :=
    exists LL,
      adj LL bot L /\
      mall LL
...
\end{lstlisting}
\end{frame}

\begin{frame}[fragile]
  \frametitle{Meta-theory: inversion}

  \judg{
    \infer[$\PAR^{-1}$]{
      \ts \G, A, B
    }{
      \ts \G, A \PAR B
    }
  }

\begin{lstlisting}
Theorem par_inv : forall G A B D,
  mall G -> adj D (tens A B) G ->
  exists U V, adj D A U /\ adj U B V /\ mall V.
\end{lstlisting}
\end{frame}

\begin{frame}[fragile]
  \frametitle{Meta-theory: cut}

  \judg{
    \smaller
    \infer{
      \ts (\G \setminus A), (\D \setminus \dual A)
    }{
      \ts \G
      &
      \qquad
      \ts \D
    }
  }

\begin{lstlisting}
Theorem cut_admit : forall A B G GG D DD W,
  dual A B ->
  mall G -> adj GG A G ->
  mall D -> adj DD B D ->
  merge GG DD W -> mall W.
induction on 1. induction on 2.
  ...
\end{lstlisting}
\end{frame}

\begin{frame}
  \frametitle{Exponentials}

  \begin{gather*}
    \linfer{
      \ts \Xi ; a, \dual a
    }{}
    \quad
    \linfer{
      \ts \Xi ; \G, \D, A \TENS B
    }{
      \ts \Xi ; \G, A
      &
      \ts \Xi ; \D, B
    }
    \quad
    \linfer{
      \ts \Xi ; \ONE
    }{}
    \quad
    \linfer{
      \ts \Xi ; \G, A \PAR B
    }{
      \ts \Xi ; \G, A, B
    }
    \quad
    \linfer{
      \ts \Xi ; \G, \BOT
    }{
      \ts \Xi ; \G
    }
    \\[2ex]
    \linfer{
      \ts \Xi ; \G, A \WITH B
    }{
      \ts \Xi ; \G, A
      &
      \ts \Xi ; \G, B
    }
    \quad
    \linfer{
      \ts \Xi ; \G, \TOP
    }{}
    \quad
    \linfer{
      \ts \Xi ; \G, A \PLUS B
    }{
      \ts \Xi ; \G, A
    }
    \quad
    \linfer{
      \ts \Xi ; \G, A \PLUS B
    }{
      \ts \Xi ; \G, B
    }
    \\
    \hbox to .8\linewidth{\dotfill}
    \\[1ex]
    \linfer{
      \ts \Xi ; \BANG A
    }{
      \ts \Xi ; A
    }
    \qquad
    \linfer{
      \ts \Xi ; \G, \QM A
    }{
      \ts \Xi, A ; \G
    }
    \qquad
    \linfer{
      \ts \Xi, A ; \G
    }{
      \ts \Xi, A ; \G, A
    }
  \end{gather*}
\end{frame}

\begin{frame}
  \frametitle{Exponentials: lexicographic cuts}

  \judg{
    \linfer[hcut]{
      \ts \Xi ; \G
    }{
      \deduce{\ts \Xi ; P}{\DD}
      &
      \ts \Xi, \dual P ; \G
    }
    \qquad
    \linfer[lcut]{
      \ts \Xi ; \G, \D
    }{
      \deduce{\ts \Xi ; \G, P}{\DD}
      &
      \ts \Xi ; \G, \dual P
    }
  }

  \quad

  The inductive hypothesis may be used whenever:
  \begin{enumerate}
  \item either the size of $P$ (= rank) is smaller; or
  \item the rank is the same and the height of $\DD$ is smaller; or
  \item a lcut is reduced to a hcut of the same rank and $\DD$-height.
  \end{enumerate}
\end{frame}

\begin{frame}[fragile]
  \frametitle{Encoding dyadic sequents}

  \mbox{}\bigskip

\begin{lstlisting}
Define mell : olist -> olist -> prop by
; mell QL L :=
    exists A, adj (natom A :: nil) (atom A) L

; mell QL L :=
    exists A QK, adj QK A QL /\
    exists J, adj L A J /\
    mell QL J

; mell QL L :=
    exists A B LL, adj LL (tens A B) L /\
    exists JJ KK, merge JJ KK LL /\
      (exists J, adj JJ A J /\ mell QL J) /\
      (exists K, adj KK B K /\ mell QL K)

; mell QL (one :: nil)

; mell QL L :=
    exists A B LL, adj LL (par A B) L /\
    exists J, adj LL A J /\
    exists K, adj J B K /\
    mell QL K

; mell QL L :=
    exists LL, adj LL bot L /\
    mell QL LL

; mell QL (bang A :: nil) :=
    mell QL (A :: nil)

; mell QL L :=
    exists A LL, adj LL (qm A) L /\
    exists QK, adj QL A QK /\
    mell QK LL.
\end{lstlisting}
\end{frame}

\begin{frame}[fragile]
  \frametitle{Meta-theory: lexicographic cuts}

\begin{lstlisting}
Kind weight type.
Type heavy, light weight.

Define is_weight : weight -> prop by
; is_weight light
; is_weight heavy := is_weight light.

Theorem cut_admit: forall A B G Wt,
   dual A B -> mell QL G -> is_weight Wt ->
      (Wt = heavy /\ G = A :: nil /\
       forall QK, adj QL B QK -> mell QK D ->
       mell QL D)
   \/ (Wt = light /\
       forall GG D DD W, adj GG A G ->
       mell QL D -> adj DD B D ->
       merge GG DD W -> mell QL W).
induction on 1. induction on 2. induction on 3.
  ...
\end{lstlisting}
\end{frame}

\begin{frame}
  \frametitle{Focusing}

  \begin{gather*}
    \linfer{
      \ts \foc{a}, \dual a
    }{}
    \qquad
    \linfer{
      \ts \G, \D, \foc{A \TENS B}
    }{
      \ts \G, \foc{A}
      &
      \ts \D, \foc{B}
    }
    \qquad
    \linfer{
      \ts \foc{\ONE}
    }{}
    \qquad
    \linfer{
      \ts \G, A \PAR B
    }{
      \ts \G, A, B
    }
    \qquad
    \linfer{
      \ts \G, \BOT
    }{
      \ts \G
    }
    \\[2ex]
    \linfer{
      \ts \G, A \WITH B
    }{
      \ts \G, A
      &
      \ts \G, B
    }
    \qquad
    \linfer{
      \ts \G, \TOP
    }{}
    \qquad
    \linfer{
      \ts \G, \foc{A \PLUS B}
    }{
      \ts \G, \foc{A}
    }
    \qquad
    \linfer{
      \ts \G, \foc{A \PLUS B}
    }{
      \ts \G, \foc{B}
    }
    \\
    \hbox to .8\linewidth{\dotfill}
    \\[1ex]
    \linfer{
      \ts \G, P
    }{
      \ts \G, \foc{P}
      &
      (\G \text{ all positive})
    }
    \qquad
    \linfer{
      \ts \G, \foc{N}
    }{
      \ts \G, N
    }
  \end{gather*}

\end{frame}

\begin{frame}
  \frametitle{Polarized focusing}

  \begin{align*}
    P, Q, \dotsc &::=
    a \mid P \TENS Q \mid \ONE \mid
    P \PLUS Q \mid \ZERO \mid \SHP N \\
    N, M, \dotsc &::=
    \dual a \mid N \PAR M \mid \BOT \mid
    N \WITH M \mid \TOP \mid \SHN P \\
    \text{neutral} &::= \dual a \mid \SHN P
  \end{align*}

  \begin{gather*}
    \linfer{
      \ts \foc{a}, \dual a
    }{}
    \qquad
    \linfer{
      \ts \G, \D, \foc{P \TENS Q}
    }{
      \ts \G, \foc{P}
      &
      \ts \G, \foc{Q}
    }
    \qquad
    \linfer{
      \ts \foc{\ONE}
    }{}
    \qquad
    \linfer{
      \ts \G, N \PAR M
    }{
      \ts \G, N, M
    }
    \qquad
    \linfer{
      \ts \G, \BOT
    }{
      \ts \G
    }
    \\[2ex]
    \linfer{
      \ts \G, N \WITH M
    }{
      \ts \G, N
      &
      \ts \G, M
    }
    \qquad
    \linfer{
      \ts \G, \TOP
    }{}
    \qquad
    \linfer{
      \ts \G, \foc{P \PLUS Q}
    }{
      \ts \G, \foc{P}
    }
    \qquad
    \linfer{
      \ts \G, \foc{P \PLUS Q}
    }{
      \ts \G, \foc{Q}
    }
    \\
    \hbox to .8\linewidth{\dotfill}
    \\[1ex]
    \linfer{
      \ts \G, \SHN P
    }{
      \ts \G, \foc{P}
      &
      (\G \text{ all neutral})
    }
    \qquad
    \linfer{
      \ts \G, \foc{\SHP N}
    }{
      \ts \G, N
    }
  \end{gather*}
\end{frame}

\begin{frame}
  \frametitle{Synthetic focusing}

  \begin{gather*}
    \linfer{
      a \in a \mathstrut
    }{}
    \qquad
    \linfer{
      \SHP N \in \SHP N
    }{}
    \qquad
    \linfer{
      \W_1, \W_2 \in P \TENS Q
    }{
      \W_1 \in P & \W_2 \in Q
    }
    \qquad
    \linfer{
      \cdot \in \ONE
    }{}
    \\[2ex]
    \linfer{
      \W \in P \PLUS Q
    }{
      \W \in P
    }
    \qquad
    \linfer{
      \W \in P \PLUS Q
    }{
      \W \in Q
    }
\onslide<2->{\alert{
    \qquad
    \linfer{
      \W \in \exists x. P
    }{
      \W \in [t/x] P
    }
}}
    \\
    \hbox to .8\linewidth{\dotfill}
    \\[1ex]
    \linfer{
      \ts \dual a : \foc{a}
    }{}
    \qquad
    \linfer{
      \ts \cdot : \foc{\cdot}
    }{}
    \qquad
    \linfer{
      \ts \G_1, \G_2 : \foc{\W_1, \W_2}
    }{
      \ts \G_1 : \foc{\W_1}
      &
      \ts \G_2 : \foc{\W_2}
    }
    \\
    \hbox to .8\linewidth{\dotfill}
    \\[1ex]
    \linfer{
      \ts \G, \SHN P
    }{
      (\W \in P) & \ts \G : \foc{\W}
    }
    \qquad
    \linfer{
      \ts \G : \foc{\SHP \dual P}
    }{
      \Bigl\{{
        \ts \G, \dual \W
      }\Bigr\}_{\mathrlap{\W \in P}}
    }
  \end{gather*}
\end{frame}

\begin{frame}[fragile]
  \frametitle{Formalizing synthetics}

\begin{lstlisting}
Kind foc type.
Type fatom  atm -> foc.
Type fshift nf -> foc.
Type fjoin  foc -> foc -> foc.
Type femp   foc.

Define subf : foc -> pf -> prop by
; subf (fatom A) (atom A)
; subf (fshift N) (shp N)
; subf (fjoin F1 F2) (tens P Q) :=
    subf F1 P /\ subf F2 Q
; subf F (oplus P Q) := subf F P
; subf F (oplus P Q) := subf F Q
; nabla x, subf (F x) (fex P) :=
    nabla x, subf (F x) (P x)
; subf femp one.
\end{lstlisting}
\end{frame}

\begin{frame}[fragile]
  \frametitle{Formalizing synthetic focusing}

  \smaller

\begin{lstlisting}
Define
  mall : olist -> prop,
  mallfoc : olist -> foc -> prop
by
; mall L :=
    exists P LL, adj LL P L /\
    exists F,    subf F P /\
                 mallfoc LL F

; mallfoc (natom A :: nil) (fatom A)

; mallfoc L (fshift N) :=
    exists P,  dual P N /\
    forall F,  subf F P ->
    exists LL, extend L F LL /\
               mall LL

; mallfoc L (fjoin F1 F2) :=
    exists J K, merge J K L /\
    mallfoc J F1 /\ mallfoc K F2

; mallfoc nil femp.
\end{lstlisting}
\end{frame}

\begin{frame}[fragile]
  \frametitle{Synthetic meta-theory}

  \judg{
    \infer={
      \ts \G, \D
    }{
      \ts \G, \SHN P
      &
      \ts \D, \SHN \SHP \dual P
    }
  }

\begin{lstlisting}[
  emph={mall},
  emphstyle=\color{abellabad}]
Theorem cut_admit : forall P N J JP K L,
   dual P N ->
   adj J (shn P) JP       -> mall JP ->
   adj K (shn (shp N)) KN -> mall KN ->
   merge J K L            -> mall L.
\end{lstlisting}
\end{frame}

\begin{frame}
  \frametitle{Synthetic meta-theory: but really!}

  Principal:
  \judg{
    \infer={
      \ts \G, \D
    }{
      (\W \in P)
      &
      \ts \G : \foc{\W}
      &
      \ts \D : \foc{\SHP \dual P}
    }
  }

  Commutative:
  \judg{
    \infer={
      \ts \G, \D : [\W]
    }{
      \ts \G, \SHN P : [\W]
      &
      \ts \D : \foc{\SHP \dual P}
    }
  }
\end{frame}

\begin{frame}
  \frametitle{Some stats}

  \centering
  \url{https://github.com/meta-logic/abella-reasoning}

  \vspace{1cm}

  \begin{tabu}{l|c|c}
    & ordinary & synthetic \\\hline
    \vrule width 0pt height 1.3em
    Prop. MALL & \alert{774} & --- \\
    Prop. MELL & 694 & --- \\
    Prop. LL & 1230 & --- \\[1ex]
    $\forall$MALL & --- & \alert{490} \\
    $\forall$LL & --- & ---
  \end{tabu}\\
  {\scriptsize (unfiltered lines of text)}
\end{frame}

\begin{frame}[fragile]
  \frametitle{Abella's induction}

  So far we have used Abella's \emph{implicit} heights in our
  inductive proofs.

  \bigskip

\begin{lstlisting}
Theorem trivial : forall x, nat x -> lt x (s x).
induction on 1. intros. case H1.
  search.
  apply IH to H2. search.
\end{lstlisting}

  \pause

\begin{lstlisting}
IH : forall x, nat x * -> lt x (s x)
============================
 forall x, nat x @ -> lt x (s x)
\end{lstlisting}

  \pause

\begin{lstlisting}
Variables: X
IH : forall x, nat x * -> lt x (s x)
H2 : nat X *
============================
 lt (s X) (s (s X))
\end{lstlisting}

\end{frame}

\begin{frame}[fragile]
  \frametitle{Explicit heights in sequents}

  \smaller[2]

\begin{lstlisting}[emph={s,H},emphstyle={\color{red}}]
Define mallh : nat -> olist -> prop by
  % init
; mallh (s H) L :=
    exists A, adj (natom A :: nil) (atom A) L

  % tensor
; mallh (s H) L :=
    exists A B LL, adj LL (tens A B) L /\
    exists JJ KK, merge JJ KK LL /\
    exists J, adj JJ A J /\ mallh H J /\
    exists K, adj KK B K /\ mallh H K.

; mallh (s H) (one :: nil)

  % par
; mallh (s H) L :=
    exists A B LL, adj LL (par A B) L /\
    exists J, adj LL A J /\
    exists K, adj J B K /\ mallh H K

; mallh (s H) L :=
    exists LL, adj LL bot L /\ mallh H LL
...
\end{lstlisting}

\end{frame}

\begin{frame}[fragile]
  \frametitle{Equivalence}

  Abella's induction is strong enough to prove the following theorems.

  {\smaller[2]
\begin{lstlisting}
Theorem equiv_ltr: forall H L, mallh H L -> mall L.

Theorem equiv_rtl: forall L, mall L -> exists H, nat H /\ mallh H L.
\end{lstlisting}}

   \bigskip \pause

   This allows us to drop down to explicit heights if we ever need
   them, but to continue without explicit heights in general.
\end{frame}

\begin{frame}
  \frametitle{A mystery}

  \begin{itemize}
  \item I initially thought that we could do the same thing for the
    synthetic version of the calculus.
  \item To my surprise:
    \begin{itemize}
    \item I couldn't show the equivalence between the two formalisms
    \item I couldn't even prove cut-elimination by itself in the
      version with explicit heights
    \end{itemize}
  \item The culprit is the following rule:
    \begin{gather*}
      \infer{
        \ts \G : \foc{\SHP \dual P}
      }{
        \Bigl\{{
          \ts \G, \dual \W
        }\Bigr\}_{\mathrlap{\W \in P}}
      }
    \end{gather*}
  \end{itemize}
\end{frame}

\begin{frame}
  \frametitle{Synthetic focusing with explicit heights}

  \smaller

  For $H$ a natural number, write\\[1ex]
  \begin{tabu}{l@{\qquad}l}
    $H \ts \G$ & neutral sequent of max. deriv. height $H$ \\
    $H \ts \G : \foc{\W}$ & focused sequent of max. deriv. height $H$
  \end{tabu}


  \begin{gather*}
    \linfer{
      a \in a \mathstrut
    }{}
    \qquad
    \linfer{
      \SHP N \in \SHP N
    }{}
    \qquad
    \linfer{
      \W_1, \W_2 \in P \TENS Q
    }{
      \W_1 \in P & \W_2 \in Q
    }
    \qquad
    \linfer{
      \cdot \in \ONE
    }{}
    \\[2ex]
    \linfer{
      \W \in P \PLUS Q
    }{
      \W \in P
    }
    \qquad
    \linfer{
      \W \in P \PLUS Q
    }{
      \W \in Q
    }
    \qquad
    \linfer{
      \W \in \exists x. P
    }{
      \W \in [t/x] P
    }
    \\
    \hbox to .8\linewidth{\dotfill}
    \\[1ex]
    \linfer{
      H \ts \dual a : \foc{a}
    }{}
    \qquad
    \linfer{
      H \ts \cdot : \foc{\cdot}
    }{}
    \qquad
    \linfer{
      H \ts \G_1, \G_2 : \foc{\W_1, \W_2}
    }{
      H \ts \G_1 : \foc{\W_1}
      &
      H \ts \G_2 : \foc{\W_2}
    }
    \\
    \hbox to .8\linewidth{\dotfill}
    \\[1ex]
    \linfer{
      (H + 1) \ts \G, \SHN P
    }{
      (\W \in P) & H \ts \G : \foc{\W}
    }
    \qquad
    \linfer{
      (H + 1) \ts \G : \foc{\SHP \dual P}
    }{
      \Bigl\{{
        H \ts \G, \dual \W
      }\Bigr\}_{\mathrlap{\W \in P}}
    }
  \end{gather*}
\end{frame}

\begin{frame}
  \frametitle{Cut rules with explicit heights}

  Principal:
  \judg{
    \infer={
      H_3 \ts \G, \D
    }{
      (\W \in P)
      &
      H_1 \ts \G : \foc{\W}
      &
      H_2 \ts \D : \foc{\SHP \dual P}
    }
  }

  Commutative:
  \judg{
    \infer={
      H_3 \ts \G, \D : [\W]
    }{
      H_1 \ts \G, \SHN P : [\W]
      &
      H_2 \ts \D : \foc{\SHP \dual P}
    }
  }

  $H_1, H_2, H_3$ unrelated, or maybe $H_1\ H_2 \ge H_3 \ge H_1 + H_2$.
\end{frame}

\begin{frame}
  \frametitle{Commutative case in detail}

  \begin{gather*}
    \infer={
      H_3 \ts \G, \D : [\W]
    }{
      H_1 \ts \G, \SHN P : [\W]
      &
      \infer{
        H_2 + 1 \ts \D : \foc{\SHP \dual P}
      }{
        \Bigl\{{
          H_2 \ts \D, \dual{\W'}
        }\Bigr\}_{\mathrlap{\W' \in P}}
      }
    }
  \end{gather*}

  \bigskip

  \begin{align*}
    IH &:\quad \forall \W' \in P.\,\ \exists H_3.\,\
    H_3 \ts \G, \D : \foc{\W} \\
    \text{goal} &:\quad
    \exists H_3'.\,\ \forall \W' \in P.\,\ H_3' \ts \G', \D, \dual{\W'}
  \end{align*}
\end{frame}

\begin{frame}
  \frametitle{Strong continuity?}

  This is reminiscent of the so called \emph{strong continuity
    principle}, which is a consequence of Brouwer's fan theorem:
  %
  \begin{gather*}
    (\forall s\in \mathrm{N}^*\mkern -2mu.\, \exists p.\, A(s, p))
    \supset
    (\exists f \in K_0.\, \forall s.\, A(s, f(s)))
  \end{gather*}
  %
  where $K_0$ is the class of all computable functions that depend on
  only a \alert{finite prefix} of its input.

  \bigskip \pause

  TODO: figure out if Abella's induction has enough strength to do
  some kinds of bar induction (cf. recent work by Vincent Rahli, Bob
  Constable, \emph{et al}), and then to see if the SCP or a suitable
  variant can be derived.
\end{frame}

\end{document}

%%% Local Variables:
%%% mode: latex
%%% TeX-engine: xetex
%%% TeX-master: t
%%% End: